\documentclass[10pt]{article}

\usepackage{geometry}
\geometry{
 a4paper,
 total={160mm,250mm},
 left=15mm,
 right=15mm,
 top=20mm,
}

\usepackage[utf8x]{inputenc}
\usepackage[pdftex]{graphicx}
\usepackage{bm}
\usepackage{import}
\usepackage{xifthen}
\usepackage{pdfpages}


\newcommand{\incfig}[3][1]{%
    \def\svgwidth{#1\columnwidth}
    \import{#2}{#3.pdf_tex}
}


\title{Importing Images from Inkscape to \LaTeX}
\author{Georgios Arampatzis\\Mathematics and Applied Mathematics Department\\University of Crete}
\date{}


\begin{document}

\maketitle



\section{Workflow: Inkscape to LaTeX}

\subsection{Creating and Saving Graphics}
To create graphics for use with the \texttt{import} package, follow these steps in
Inkscape:

\begin{enumerate}
    \item \textbf{Design:} Create your vector graphic. Use the text tool for any
    labels you want \LaTeX{} to typeset.
    \item \textbf{Save SVG:} Navigate to \textbf{File \textgreater{} Save As...} and
    select \textbf{Inkscape SVG (*.svg)}.
    \item \textbf{Export PDF + LaTeX:}
    \begin{itemize}
        \item Go to \textbf{File \textgreater{} Export...} (or \texttt{Ctrl+Shift+E}).
        \item Select \textbf{Portable Document Format (*.pdf)} from the format
        dropdown.
        \item In the export settings, check the box: \textbf{Omit text in PDF and
        create LaTeX file}.
        \item Click \textbf{Export}.
    \end{itemize}
\end{enumerate}



Inkscape generates two files: \texttt{filename.pdf} (graphics) and
\texttt{filename.pdf\_tex} (text positioning).

\subsection{Importing into LaTeX}

You need \texttt{\textbackslash usepackage\{import, graphicx, xcolor\}}.
Include the graphic as follows:

\begin{verbatim}
\begin{figure}[ht]
    \centering
    \incfig[0.8]{./figures/}{filename}
    \caption{Description of the vector graphic.}
    \label{fig:my_graphic}
\end{figure}
\end{verbatim}




\begin{figure}[ht]
    \centering
    \incfig[0.75]{figures/histogram/}{histogram}
    \caption{
        Histogram using relative frequencies.
        }
    \label{fig:age:histogram}
\end{figure}



\begin{figure}[ht]
    \centering
    \incfig[0.75]{figures/mode/}{interp}
    \caption{
        Mode interpolation at the points $(m_{j-1}, f_{j-1}), (m_{j}, f_{j}), (m_{j+1}, f_{j+1})$ we interpolate a second-degree polynomial, $p$.
        We assume that $p$ reaches its maximum at $m \in [a_{j-1}, a_j]$.
        }
    \label{fig:mode_interpolation}
\end{figure}


\begin{figure}[ht]
    \centering
    \incfig[0.75]{figures/mode/}{fig}
    \caption{
        The first derivative of a second-degree polynomial is a linear function.
        The derivative vanishes at $m$ and the unknown is the distance $h$ from the end $a_j$ of the class $C_j$.
    }
    \label{fig:mode_first_derivative}
\end{figure}


\begin{figure}[ht]
    \centering
    \incfig[0.75]{figures/box-plot/}{boxplot}
    \caption{
        Box plot.
    }
    \label{fig:boxplot}
\end{figure}


\begin{figure}[ht]
    \centering
    \incfig[0.9]{figures/asymmetry/}{asymmetry}
    \caption{
        Distributions with positive asymmetry, symmetry, and negative asymmetry.
    }
    \label{fig:assymetry}
\end{figure}


\begin{figure}[ht]
    \centering
    \incfig[1]{figures/prisoners/}{prisoners}
    \caption{
        The prisoners' problem for 8 prisoners. An arrangement of the numbers 1 to 8 in which the prisoners win. 
        Prisoner number 1 opens box 1, then box 2, then box 3, and stops because he finds his number. 
        Prisoner number 8 opens box 8, then box 6, then box 5, then box 5 again, and stops because he finds his number. 
        All cycles have a length less than or equal to 4.
        }
    \label{fig:prisoners}
\end{figure}


\begin{figure}[ht]
    \centering
    \incfig[0.7]{figures/regression/}{regression}
    \caption{
        A set of points $\{(x_i, y_i)\}_{i=1}^n$ and $\hat{y}_i$ the prediction of the model at the point $x_i$.
        }
    \label{fig:regression}
\end{figure}


\begin{figure}[ht]
    \centering
    \incfig[0.7]{figures/nullspace/}{nullspace}
    \caption{
        The data vector $\bm{y}$ is the sum of two vectors:
        the prediction vector $\hat{\bm{y}}$, which is a linear
        combination of the columns of the matrix $A$,
        and the error vector $\bm{\varepsilon}$, which is orthogonal to $\hat{\bm{y}}$.
        The vector $\hat{\bm{y}}$ is the projection of $\bm{y}$ onto the column space via
        the projection matrix $P$, and the vector $\bm{\varepsilon}$ is the projection
        of $\bm{y}$ onto the nullspace of the column space via the projection matrix $I-P$.
        }
    \label{fig:n=ullspace}
\end{figure}




\end{document}
